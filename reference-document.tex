\documentclass{article}
\usepackage[utf8]{inputenc}
% \usepackage[russian]{babel}
\usepackage[a4paper, top=0.7in, left=0.5in, right=0.5in, bottom=0.6in, twocolumn]{geometry}
\usepackage{lastpage}
\usepackage{fancyhdr}
\usepackage{tikz}
\usepackage{pgfplots}
\usepackage{amssymb}
\usepackage{minted}
\usepackage{pdfpages}
\usepackage{booktabs}
\usepackage{hyperref}
\usepackage{amsmath}
\usepackage{titlesec}

\usetikzlibrary{shapes}

\setcounter{secnumdepth}{5}
\setcounter{tocdepth}{5}
\pgfkeys{/pgf/number format/.cd,1000 sep={\,}}

\titlespacing*{\section}{0pt}{0ex}{0ex}
\titlespacing*{\subsection}{0pt}{0ex}{0ex}

\pagestyle{fancy}
\fancyhf{}
\lhead{ITMO University 1: Insert your name (Budin, Korobkov, Naumov)}
\rhead{Page \thepage\ of \pageref{LastPage}}
\lfoot{Generated \today}

\renewcommand{\footrulewidth}{0.4pt}
\setlength{\columnseprule}{0.4pt}

\begin{document}
% \onecolumn
\tableofcontents

% \twocolumn
\newpage

\section{Some useful stuff}
\subsection{Fast I/O}
\inputminted[mathescape, breaklines, breakafter=(, tabsize=2, frame=lines, showtabs, tab=|\ , tabcolor=lightgray]{c++}{./basic/fast-io/io.cpp}
\subsection{Pragmas}
\inputminted[mathescape, breaklines, breakafter=(, tabsize=2, frame=lines, showtabs, tab=|\ , tabcolor=lightgray]{c++}{./basic/pragmas/opt.cpp}
\section{Data structures}
\subsection{Cartesian Tree}
\inputminted[mathescape, breaklines, breakafter=(, tabsize=2, frame=lines, showtabs, tab=|\ , tabcolor=lightgray]{c++}{./data-structures/cartesian-tree/cartesian-tree.cpp}
\subsection{Dynamic convex hull trick}
\inputminted[mathescape, breaklines, breakafter=(, tabsize=2, frame=lines, showtabs, tab=|\ , tabcolor=lightgray]{c++}{./data-structures/convex-hull-trick/convex-hull-trick.cpp}
\subsection{Fenwick tree}
\inputminted[mathescape, breaklines, breakafter=(, tabsize=2, frame=lines, showtabs, tab=|\ , tabcolor=lightgray]{c++}{./data-structures/fenwick/fenwick.cpp}
\subsection{Hash table}
\inputminted[mathescape, breaklines, breakafter=(, tabsize=2, frame=lines, showtabs, tab=|\ , tabcolor=lightgray]{c++}{./data-structures/hash-table/hash-table.cpp}
\subsection{Persistent Segment Tree}
\inputminted[mathescape, breaklines, breakafter=(, tabsize=2, frame=lines, showtabs, tab=|\ , tabcolor=lightgray]{c++}{./data-structures/persistent-segment-tree/persistent-segment-tree..cpp}
\subsection{Ordered set and bitset}
\inputminted[mathescape, breaklines, breakafter=(, tabsize=2, frame=lines, showtabs, tab=|\ , tabcolor=lightgray]{c++}{./data-structures/std/std.cpp}
\section{Geometry}
\subsection{Common tangents of two circles}
\inputminted[mathescape, breaklines, breakafter=(, tabsize=2, frame=lines, showtabs, tab=|\ , tabcolor=lightgray]{c++}{./geometry/common-tangents/common-tangents.cpp}
\subsection{Convex hull 3D in $O(n ^ 2)$}
\inputminted[mathescape, breaklines, breakafter=(, tabsize=2, frame=lines, showtabs, tab=|\ , tabcolor=lightgray]{c++}{./geometry/convex-hull-3d/convex-hull-3d.cpp}
\subsection{Minimal covering disk}
\inputminted[mathescape, breaklines, breakafter=(, tabsize=2, frame=lines, showtabs, tab=|\ , tabcolor=lightgray]{c++}{./geometry/min-disk/min-disk.cpp}
\subsection{Polygon tangent}
\inputminted[mathescape, breaklines, breakafter=(, tabsize=2, frame=lines, showtabs, tab=|\ , tabcolor=lightgray]{c++}{./geometry/polygon-tangent/polygon-tangent.cpp}
\subsection{Rotate 3D}
\inputminted[mathescape, breaklines, breakafter=(, tabsize=2, frame=lines, showtabs, tab=|\ , tabcolor=lightgray]{c++}{./geometry/rotate-3d/rotate-3d.cpp}
\subsection{Rotation matrix 2D}
Rotation of point $(x, y)$ through an angle $\alpha$ in counterclockwise direction in 2D.

$$
\begin{pmatrix}
\cos \alpha & -\sin \alpha \\
\sin \alpha & \cos \alpha
\end{pmatrix}
\cdot
\begin{pmatrix}
x \\
y
\end{pmatrix}
=
\begin{pmatrix}
x' \\
y'
\end{pmatrix}
$$
\subsection{Sphere distance}
\inputminted[mathescape, breaklines, breakafter=(, tabsize=2, frame=lines, showtabs, tab=|\ , tabcolor=lightgray]{c++}{./geometry/sphere-dist/sphere-dist.cpp}
\subsection{Draw svg pictures}
\inputminted[mathescape, breaklines, breakafter=(, tabsize=2, frame=lines, showtabs, tab=|\ , tabcolor=lightgray]{c++}{./geometry/svg-draw/svg-draw.cpp}
\section{Graphs}
\subsection{General matching}
\inputminted[mathescape, breaklines, breakafter=(, tabsize=2, frame=lines, showtabs, tab=|\ , tabcolor=lightgray]{c++}{./graphs/general-matching/general-matching.cpp}
\subsection{Hungarian algorithm}
\inputminted[mathescape, breaklines, breakafter=(, tabsize=2, frame=lines, showtabs, tab=|\ , tabcolor=lightgray]{c++}{./graphs/hungarian-algorithm/hungarian-algorithm.cpp}
\subsection{Kuhn and Min Vertex Covering}
\inputminted[mathescape, breaklines, breakafter=(, tabsize=2, frame=lines, showtabs, tab=|\ , tabcolor=lightgray]{c++}{./graphs/kuhn-and-min-covering/kuhn-and-min-covering.cpp}
\section{Numeric}
\subsection{Burnside's lemma}
$$|X/G| = \frac{1}{|G|}\sum\limits_{g \in G}|St(g)|$$

$St(g)$ denote the set of elements in $X$ that are fixed by $g$, i.e. $St(g) = \{x \in X | gx = x\}$.
\subsection{Chinese remainder theorem}
\inputminted[mathescape, breaklines, breakafter=(, tabsize=2, frame=lines, showtabs, tab=|\ , tabcolor=lightgray]{c++}{./numeric/chinese-remainder-theorem/chinese-remainder-theorem.cpp}
\subsection{AND/OR/XOR convolution}
\inputminted[mathescape, breaklines, breakafter=(, tabsize=2, frame=lines, showtabs, tab=|\ , tabcolor=lightgray]{c++}{./numeric/convolutions/convolutions.cpp}
\subsection{Counting size of the maximum general matching}
In order to find a size of the maximum matching:
\begin{enumerate}
	\item Build Tutte matrix. ($x_{ij}$ are random numbers)
	$$A_{ij} = 
	\begin{cases} 
	x_{ij} & \text{if edge $(i, j)$ exists and $i < j$} \\
	-x_{ij} & \text{if edge $(i, j)$ exists and $i > j$} \\
	0 & otherwise
	\end{cases}$$
	\item The size of the maximum matching equals to the size of the maximum independent set divided by $2$.
	\item $(A^{-1})_{ji} \neq 0$ iff edge $(i, j)$ belongs to some complete matching.
\end{enumerate}
\subsection{Counting number of spanning trees}
In order to count number of spanning trees:
\begin{enumerate}
	\item Build the Laplacian matrix. That is difference between the degree matrix and the adjacency matrix.
	\item Delete any row and any column of this matrix.
	\item Calculate it's determinant.
\end{enumerate}
\subsection{Some formulas}
\begin{itemize}
\item $\sum\limits_{i = 1}^{n} i^2 = \frac{n(n + 1)(2n + 1)}{6}$
\item $\sum\limits_{i = 1}^{n} i^3 = \frac{n^2(n + 1)^2}{4}$
\item $\sum\limits_{i = 1}^{n} i^4 = \frac{n(n + 1)(2n + 1)(3n^2 + 3n - 1)}{30}$
\item $\sum\limits_{k = 0}^{n} k \binom{n}{k} = n 2^{n - 1}$
\item $\sum\limits_{k = 0}^{n} \binom{n}{k}^2 = \binom{2n}{n}$
\end{itemize}
\subsection{Miller–Rabin primality test}
\inputminted[mathescape, breaklines, breakafter=(, tabsize=2, frame=lines, showtabs, tab=|\ , tabcolor=lightgray]{c++}{./numeric/miller-rabin/miller-rabin.cpp}
\subsection{Taking by modullo (Inline assembler)}
\inputminted[mathescape, breaklines, breakafter=(, tabsize=2, frame=lines, showtabs, tab=|\ , tabcolor=lightgray]{c++}{./numeric/mod-asm/mod-asm.cpp}
\subsection{First solution of $(p + step \cdot x) \bmod mod < l$}
\inputminted[mathescape, breaklines, breakafter=(, tabsize=2, frame=lines, showtabs, tab=|\ , tabcolor=lightgray]{c++}{./numeric/mod-ineq-first-sol/mod-ineq-first-sol.cpp}
\subsection{Multiplication by modulo in \texttt{long double}}
\inputminted[mathescape, breaklines, breakafter=(, tabsize=2, frame=lines, showtabs, tab=|\ , tabcolor=lightgray]{c++}{./numeric/mult-by-mod/mult-by-mod.cpp}
\subsection{Numerical integration}
\inputminted[mathescape, breaklines, breakafter=(, tabsize=2, frame=lines, showtabs, tab=|\ , tabcolor=lightgray]{c++}{./numeric/numerical-integration/numerical-integration.cpp}
\subsection{Pollard's rho algorithm}
\inputminted[mathescape, breaklines, breakafter=(, tabsize=2, frame=lines, showtabs, tab=|\ , tabcolor=lightgray]{c++}{./numeric/pollard/pollard.cpp}
\subsection{Polynom division and inversion}
\inputminted[mathescape, breaklines, breakafter=(, tabsize=2, frame=lines, showtabs, tab=|\ , tabcolor=lightgray]{c++}{./numeric/polynom-division/polynom-division.cpp}
\subsection{Polynom roots}
\inputminted[mathescape, breaklines, breakafter=(, tabsize=2, frame=lines, showtabs, tab=|\ , tabcolor=lightgray]{c++}{./numeric/polynom-roots/polynom-roots.cpp}
\subsection{Simplex method}
\inputminted[mathescape, breaklines, breakafter=(, tabsize=2, frame=lines, showtabs, tab=|\ , tabcolor=lightgray]{c++}{./numeric/simplex/simplex.cpp}
\subsection{Some integer sequences}
\begin{center}
\begin{tabular}{|r|r|r|r|}
\hline
\multicolumn{4}{|l|}{Bell numbers:} \\
\hline
$n$ & $B_n$ & $n$ & $B_n$ \\
\hline
$0$ & $1$ & $10$ & $115\,975$ \\
\hline
$1$ & $1$ & $11$ & $678\,570$ \\
\hline
$2$ & $2$ & $12$ & $4\,213\,597$ \\
\hline
$3$ & $5$ & $13$ & $27\,644\,437$ \\
\hline
$4$ & $15$ & $14$ & $190\,899\,322$ \\
\hline
$5$ & $52$ & $15$ & $1\,382\,958\,545$ \\
\hline
$6$ & $203$ & $16$ & $10\,480\,142\,147$ \\
\hline
$7$ & $877$ & $17$ & $82\,864\,869\,804$ \\
\hline
$8$ & $4\,140$ & $18$ & $682\,076\,806\,159$ \\
\hline
$9$ & $21\,147$ & $19$ & $5\,832\,742\,205\,057$\\
\hline
\end{tabular}
\end{center}











\begin{center}
\begin{tabular}{|r|r|r|}
\hline
\multicolumn{3}{|l|}{Numbers with many divisors:} \\
\hline
$x \le$ & $x$ & $d(x)$ \\
\hline
$20$ & $12$ & $6$ \\
\hline
$50$ & $48$ & $10$ \\
\hline
$100$ & $60$ & $12$ \\
\hline
$1000$ & $840$ & $32$ \\
\hline
$10\,000$ & $9\,240$ & $64$ \\
\hline
$100\,000$ & $83\,160$ & $128$ \\
\hline
$10^6$ & $720\,720$ & $240$ \\
\hline
$10^7$ & $8\,648\,640$ & $448$ \\
\hline
$10^8$ & $91\,891\,800$ & $768$ \\
\hline
$10^9$ & $931\,170\,240$ & $1\,344$ \\
\hline
$10^{11}$ & $97\,772\,875\,200$ & $4\,032$ \\
\hline
$10^{12}$ & $963\,761\,198\,400$ & $6\,720$ \\
\hline
$10^{15}$ & $866\,421\,317\,361\,600$ & $26\,880$ \\
\hline
$10^{18}$ & $897\,612\,484\,786\,617\,600$ & $103\,680$ \\
\hline
\end{tabular}
\end{center}
\begin{center}
\begin{tabular}{|r|r|r|r|r|r|}
\hline
\multicolumn{6}{|l|}{Partitions of $n$ into unordered summands} \\
\hline
$n$ & $a(n)$ & $n$ & $a(n)$ & $n$ & $a(n)$ \\
\hline
$0$ & $1$ & $20$ & $627$ & $40$ & $37\,338$ \\
\hline
$1$ & $1$ & $21$ & $792$ & $41$ & $44\,583$ \\
\hline
$2$ & $2$ & $22$ & $1\,002$ & $42$ & $53\,174$ \\
\hline
$3$ & $3$ & $23$ & $1\,255$ & $43$ & $63\,261$ \\
\hline
$4$ & $5$ & $24$ & $1\,575$ & $44$ & $75\,175$ \\
\hline
$5$ & $7$ & $25$ & $1\,958$ & $45$ & $89\,134$ \\
\hline
$6$ & $11$ & $26$ & $2\,436$ & $46$ & $105\,558$ \\
\hline
$7$ & $15$ & $27$ & $3\,010$ & $47$ & $124\,754$ \\
\hline
$8$ & $22$ & $28$ & $3\,718$ & $48$ & $147\,273$ \\
\hline
$9$ & $30$ & $29$ & $4\,565$ & $49$ & $173\,525$ \\
\hline
$10$ & $42$ & $30$ & $5\,604$ & $50$ & $204\,226$ \\
\hline
$11$ & $56$ & $31$ & $6\,842$ & $51$ & $239\,943$ \\
\hline
$12$ & $77$ & $32$ & $8\,349$ & $52$ & $281\,589$ \\
\hline
$13$ & $101$ & $33$ & $10\,143$ & $53$ & $329\,931$ \\
\hline
$14$ & $135$ & $34$ & $12\,310$ & $54$ & $386\,155$ \\
\hline
$15$ & $176$ & $35$ & $14\,883$ & $55$ & $451\,276$ \\
\hline
$16$ & $231$ & $36$ & $17\,977$ & $56$ & $526\,823$ \\
\hline
$17$ & $297$ & $37$ & $21\,637$ & $57$ & $614\,154$ \\
\hline
$18$ & $385$ & $38$ & $26\,015$ & $58$ & $715\,220$ \\
\hline
$19$ & $490$ & $39$ & $31\,185$ & $59$ & $831\,820$ \\
\hline
$100$ & \multicolumn{5}{|l|}{$190\,569\,292$} \\
\hline
\end{tabular}
\end{center}
\section{Strings}
\subsection{Manacher's algorithm}
\inputminted[mathescape, breaklines, breakafter=(, tabsize=2, frame=lines, showtabs, tab=|\ , tabcolor=lightgray]{c++}{./strings/manacher/manacher.cpp}
\subsection{Min Cyclic Shift O(n)}
\inputminted[mathescape, breaklines, breakafter=(, tabsize=2, frame=lines, showtabs, tab=|\ , tabcolor=lightgray]{c++}{./strings/min-cyclic_shift/min-cyclic-shift.cpp}
\subsection{Suffix array + LCP}
\inputminted[mathescape, breaklines, breakafter=(, tabsize=2, frame=lines, showtabs, tab=|\ , tabcolor=lightgray]{c++}{./strings/suff-array/suff-array.cpp}


\onecolumn
\includepdf[pages={1, 2}, pagecommand={\pagestyle{fancy}}]{integral-table}

% \begin{tikzpicture} [hexa/.style= {shape=regular polygon,
%                                    regular polygon sides=6,
%                                    minimum size=0.7cm, draw=gray,
%                                    inner sep=0, anchor=south,
%                                    fill=white}]

% \foreach \j in {0,...,32}{% 
%     \ifodd\j 
%          \foreach \i in {0,...,42}{\node[hexa] (h\j;\i) at ({(\j/2+\j/4) * 0.7},{(\i+1/2)*sin(60) * 0.7}) {};}        
%     \else
%          \foreach \i in {0,...,42}{\node[hexa] (h\j;\i) at ({(\j/2+\j/4) * 0.7},{\i*sin(60) * 0.7}) {};}
%     \fi}
% \end{tikzpicture}

\end{document}
